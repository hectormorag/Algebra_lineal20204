\documentclass{article}
\usepackage[utf8]{inputenc}
\usepackage{amsmath,amssymb}
\usepackage{systeme}
\usepackage[spanish]{babel}

\title{Álgebra Lineal 2020-4\\ Tarea 3}
\date{ }

\begin{document}

\maketitle
Estos son los enunciados para la tarea en equipo. Debes entregar la tarea con todas las indicaciones precisadas en Moodle.

\begin{enumerate}
    \item Aplica el proceso de Gram-Schmidt a la base de $\mathbb{R}^4$ que consiste de los vectores $(1,0,0,0)$, $(1,1,0,0)$, $(1,1,1,0)$ y $(1,1,1,1)$. Expresa a $(1,2,3,4)$ como combinación lineal de la base ortonormal que obtengas.
    \item Sean $r_0<r_1<r_2<\ldots<r_n$ números reales. Para cada $i$, considera la forma lineal $\text{ev}_{r_i}:\mathbb{R}_n[x]\to \mathbb{R}$ que a cada polinomio $p(x)$ lo manda a su evaluación en $r_i$, es decir, $$\text{ev}_{r_i}(p(x))=p(r_i).$$
    
    Muestra que $\text{ev}_{r_0},\text{ev}_{r_1},\ldots, \text{ev}_{r_n}$ forman una base del espacio dual  $(\mathbb{R}_n[x])^{\ast}$.
    \item Demuestra que si $\alpha_1,\ldots,\alpha_n$ son reales y $l_1,l_2,l_3$ son formas lineales en $\mathbb{R}^3$, entonces la función $$q(x)=\alpha_1l_1(x)l_2(x)+\alpha_2l_2(x)l_3(x)+\alpha_3 l_3(x)l_1(x)$$ es una forma cuadrática. Encuentra su forma polar.
    
    \item Considera $V=\mathbb{R}_3[x]$ el espacio vectorial de polinomios con coeficientes reales y grado a lo más $3$. Definimos $$\langle p,q \rangle = \sum_{j=1}^5 p(j)q(j).$$
    
    \begin{enumerate}
        \item Muestra que $\langle \cdot, \cdot \rangle$ así definido es un producto interior.
        \item Encuentra el ángulo entre los polinomios $1+x^3$ y $3x-2x^2$.
        \item Para cada entero positivo $n$, determina la norma del polinomio $1+nx^3$.
        \item Determina la distancia entre los polinomios $1$ y $1+x+x^2+x^3$.
    \end{enumerate}
    \item Sea $f:[0,1]\to \mathbb{R}^+$ una función continua que no toma valores negativos y sea $$x_n=\int_0 ^1 t^n f(t)\, dt.$$
    Demuestra que para cualesquiera $n,p\geq 0$ se tiene que 
    $$x_{n+p}\leq \sqrt{x_{2n}}\cdot \sqrt{x_{2p}}.$$ \textbf{Sugerencia:} Define un producto interior sobre el cual puedas usar la desigualdad de Cauchy-Schwarz.
\end{enumerate}

\end{document}
